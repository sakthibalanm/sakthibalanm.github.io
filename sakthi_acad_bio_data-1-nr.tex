\documentclass[11pt]{article}
\usepackage{fullpage}
\renewcommand{\baselinestretch}{1}
\begin{document}
\thispagestyle{empty}
\noindent {\bf \Large M. Sakthi Balan} \ \\\\
\noindent {\bf \large Personal Information} 
\begin{tabbing}111111111111111111111\=111\=111111111111111111 \kill
Date of birth \>:\> September 08, 1974\\
Sex \>:\> Male\\
Marital Status\>:\>Married\\
Nationality\>:\>India\\
Status in Canada\>:\>On Work Permit\\
Office Address\>:\>Department of Computer Science, University of
Western Ontario\\
\>\> London, Ontario, Canada N6A 5B7\\
Home Address\>:\> 459, Platts Lane, Apartment No. 22, London, Ontario\\
\>\> Canada N6G 3H2\\
Phone\>:\> 519 432 0998 (Home) 519 661 2111 Ext 86278 (Office)\\
E-mail\>:\> \verb+sakthi@csd.uwo.ca+\\
%      \> \> sakthi\_iitm2003@yahoo.com\\
Homepage \>:\> \verb+http://www.csd.uwo.ca/~sakthi+ 
\end{tabbing}

\vspace{0.5cm}

\noindent {\large \bf Career Objective}\\

My primary ambition is to do research and teaching in the field of computer science. 

\vspace{0.5cm}

\noindent {\large \bf Academic Background} \ \\\\
\begin{tabular}{|c|c|c|c|} \hline
Degree/Position & Institute/University & Year & Grade \\ \hline
Postdoctoral Fellow & University of Western Ontario,& April 2005 - till date & \\
& London, Ontario, Canada & & \\ \hline
Visiting Researcher & University of Western Ontario,& August 2004 -  & \\
& London, Ontario, Canada & March 2005 & \\ \hline
PhD in Computer Science & Indian Institute of Technology,& 2004 & 8.4/10 \\
and Engineering under Infosys & Madras, Chennai - 600036 &  & \\
Fellowship & India & &\\ \hline
Master of Science(by research)& Indian Institute of Technology,& 2000
& 8.4/10 \\
in Computer Science and & Madras, Chennai - 600036 & & \\
Engineering & India & & \\ \hline
M.Sc. (Mathematics)& Manonmaniam Sundaranar & 1997 & 86\% \\
& University, Tirunelveli & & \\ 
& Tamil Nadu, India & & \\ \hline
B.Sc. (Mathematics)& Manonmaniam Sundaranar & 1995 & 87\% \\
& University, Tirunelveli & & \\ 
& Tamil Nadu, India & & \\ \hline
All India Senior School & Sri Jayendra Swamigal &  1992 & 80\% \\
Certificate Examination & Silver Jubilee School & & \\
& Tirunelveli & & \\ \hline
All India Secondary & Sri Jayendra Swamigal &  1990 & 83\% \\
School Examination & Silver Jubilee School & & \\
& Tirunelveli & & \\ \hline
\end{tabular}

\vspace{0.5cm}
 
\noindent {\large \bf Academic Achievements}

\begin{itemize}
\item Reviewed research papers for 
\begin{itemize}
\item Journal of Theoretical Computer Science (TCS).
\item International Journal of Unconventional Computing.
\item 11$^{th}$ International Meeting on DNA Computing. 
\end{itemize}
\item Recipient of Infosys Technologies, India fellowship for doing
PhD in the Department of Computer Science and Engineering, Indian
Institute of Technology, Madras, 2000-2004.
\item Recipient of Indian Institute of Technology (GATE) Scholarship
for doing Masters in the Department of Computer Science and
Engineering, Indian Institute of Technology, Madras, 1998-2000. GATE
All India Rank - 105.
\item Recipient of Best Outgoing Post-Graduate student award in M.Sc Mathematics.
\item Recipient of St. Xavier's college scholarship for needy and bright students in 1995-1997.
\item Prizewinner for holding rank first in M.Sc, Mathematics.
\item Prizewinner for holding rank three in B.Sc, Mathematics.
\end{itemize}

\noindent {\large \bf Teaching Experience} 

\vspace{0.1cm}

\begin{itemize}
\item Teaching Assistant (includes occasional lectures too) at the
Department of Computer Science and Engineering, IIT Madras from
1998-2004 (during Masters and Doctorate) for the following
(under-)graduate courses. The duties for the following include
part-time teaching, preparing assignments, conducting periodical
tutorial sessions for students and evaluating students.
\begin{enumerate}
\item Unconventional Models of Computing (graduate course)

The course includes occasional teaching of various topics from
Molecular computing, preparing and evaluating assignments, and
evaluating final exams. 

\item Logic, Machines and Computations (under-graduate course)

It is a basic introductory course for computer science students which
includes occasional teaching of topics from formal languages theory and
computing theory, arranging tutorial sessions for students, preparing
assignments and evaluating, and conducting final exams and evaluating
the same.


\item Advanced Topics in Formal Language Theory (graduate course)

It is a very advanced course on various topics like grammar systems,
machine models based on molecular computing. My duties include
occasional lectures on grammar systems, preparing assignments and
evaluating final examinations.


\item Formal Languages and Automata Theory (under-graduate
  course)

It is a very basic level course for which my duties include preparing
and conducting periodical tutorials, evaluating assignments and final
examinations.


\item Mathematical Foundations of Computer Science (graduate course)

It is a basic course for graduate students -- the duties include
preparing and conducting periodical tutorial sessions for students,
evaluating assignments, conducting and evaluating examinations.

\item Discrete Mathematics (under-graduate course)

It is a basic course on mathematics for computer science students. My
duties for this course includes periodical tutorial sessions,
conducting and evaluating assignments. 

\end{enumerate}
\end{itemize}

\vspace{0.1cm}

\noindent {\large \bf Academic Visits}

\vspace{0.1cm}

\begin{itemize}
\item Visited University of York, UK for presenting paper {\it Peptide
  Computing: Universality and Theoretical Model} in Unconventional
  Computation, held from Sep 4--8 2006.
\item Visited EPFL, Lausanne, Switzerland for presenting paper {\it
Parallel Computation of Simple Arithmetic using Peptide Antibody
Interactions} in the International Workshop in Information processing
in Cells and Tissues, held from September 8--11, 2003 (visit partially
supported by Department of Science and Technology, India, Indian
National Science Academy, India and contingency grant from Infosys
Technologies, India).
\item Visiting researcher in University of Western Ontario, London,
Ontario, Canada for one month (Aug-Sep, 2002) for research discussions
in the Department of Computer Science, University of Western Ontario,
Canada and for presenting paper {\it Complexity Issues in
Binding-Blocking Automata} in the Descriptional Complexity of Formal
Systems held in the University of Western Ontario, Canada from August
21--24, 2002 (visit sponsored by University of Western Ontario).
\item Visited University of South Florida, Florida, US for presenting
paper {\it Peptide Computing Universality and Complexity} co-authored
with Prof. Kamala Krithivasan and Y.Sivasubramanyam in The Seventh
International Conference on DNA Based Computers, held from June
10--13, 2001 (visit partially supported by CSIR, India and contingency
grant from Infosys Technologies, India).
\end{itemize}


\noindent {\large \bf Areas of Interest}
\begin{itemize}
\item Molecular Computing
\item Formal Language and Automata Theory
\item Immunity-Based Systems
%\item Theory of Codes and its association with Molecular Computing
\end{itemize}

\vspace{0.3cm}

\noindent {\bf PhD Thesis Title: }Computational Models using
Peptide-Antibody Interactions

\vspace{0.4cm}

\noindent {\bf Masters Thesis Title: }Distributed Processing in Automata 

\vspace{0.4cm}

\noindent {\large \bf Courses Done}
\begin{itemize}
\item High Performance Computing
\item Database Management Systems
\item E-Commerce
\item Advanced Topics in Formal Language Theory
\item Data Structures and Programming
\item Computer Organization
\item Mathematical Foundations of Computer Science
\item Computational Geometry
\item Formal Language Theory
\item Algorithmic Graph Theory
\item Design and Analysis of Algorithm
\end{itemize}
\vspace{0.4cm}

\noindent {\large \bf Computer Practical Skills}
\begin{itemize}
\item Programming languages: C, C++, Java.
\item Operating Systems: Unix, Linux, Win9x.
\item Documentation Software: \LaTeX .
\end{itemize}

\vspace{0.4cm}

\noindent {\large \bf Mini Projects done}
\begin{itemize}
\item Implementation of Point Location Problem in Java (for
    Computational Geometry Course)
\item Writing a small search engine (for Database and Management
    Systems Course)
\item Implementation of some Graph Algorithms in C (for Algorithmic Graph Theory Course)
\item Writeup on "Virtual Carnatic Music University" (for the E-Commerce
    Course)
    \[http://nsl.cs.iitm.ernet.in/cs648/2001/2001/2001/version1/cs00p07/\]
\item Project writeup on "Dimension Theory of Posets using Hypergraph
    Coloring" (in M.Sc)
\end{itemize}

\vspace{0.4cm}

\noindent {\large \bf List of Publications}

\vspace{0.2cm}

{\large \bf Journal/Special Volumes Publications}

\begin{enumerate}
\item M. Sakthi Balan, H. J\"{u}rgensen, On the Universality of
  Peptide Computing, Natural Computing, accepted.
\item M. Sakthi Balan, Complexity Measures for Binding-Blocking
Automata, Journal of Automata, Languages and Combinatorics, accepted.
\item M. Sakthi Balan, K. Krithivasan, Parallel Computation of Simple
Arithmetic Using Peptide-Antibody Interactions, Bio-Systems, Vol. 76,
No. 1-3, pp. 303-307, 2004.
\item M. Sakthi Balan, K. Krithivasan, Realizing Switching Functions
using Peptide-Antibody Interactions, Aspects of Molecular Computing,
Lecture Notes in Computer Science, Vol. 2950, ed. N. Jonoska,
Gh. Paun, G. Rozenberg, pp. 353-360, 2004.
\item M. Sakthi Balan, Kamala Krithivasan and Mutyam Madhu, Some
variants in Communication of Parallel Communicating Pushdown Automata,
Journal of Automata, Languages and Combinatorics, Vol. 8, No. 3,
pp. 401-416, 2003.
\item K. Krithivasan, M. Sakthi Balan and R. Rama, Array Contextual
Grammars, In Recent Topics in Mathematical and Computational
Linguistics, ed. C.Martin-Vide and Gh. Paun, pp. 154-168, 2000.
\item K. Krithivasan, M. Sakthi Balan and P. Harsha, Distributed
Processing in Automata, International Journal of Foundations of
Computer Science, Vol. 10, No. 4, 443-464, 1999.
\end{enumerate}

\vspace{0.2cm}

{\large \bf Conference Publications}

\begin{enumerate}
\item M. Sakthi Balan, H. J\"{u}rgensen, Peptide Computing:
  Universality and Theoretical Model, Unconventional Computation, LNCS
  4135, pp. 57--71, 2006. (Also in a Tech report -- see below)
 \item M. Sakthi Balan, H. J\"{u}rgensen and Kamala Krithivasan,
   Peptide Computing: A Survey, Research Level Discussion on Natural
   Computing, IIT Madras, India, Nov 2005. (Also in a Tech Report --
   see below)
\item M. Sakthi Balan, K. Krithivasan, Modeling Boolean Circuits using
Peptide-Antibody Interactions, In Mathematical Biology, ed.  Peeyush
Chandra, Anshan publishers, pp. 187-193, Nov 2005.
\item M. Sakthi Balan, Algorithms for Peptide Computer, National
Conference on Algorithms and Artificial Systems, ed. P. Thangavel,
Allied Publishers, pp. 73--85, 2003.
\item M. Sakthi Balan, String Binding-Blocking Automata, Genetic and
Evolutionary Computation Conference, USA, LNCS 2723, pp. 425--426,
2003. 
\item M. Sakthi Balan, K. Krithivasan, Parallel Computation of Simple
Arithmetic Using Peptide-Antibody Interactions, International Workshop
on Information Processing in Cells and Tissues, Switzerland,
pp. 461--469, 2003.
\item M. Sakthi Balan, K.Krithivasan and Y.Sivasubramanyam, Peptide
Computing: Universality and Complexity, In N. Jonoska and N. Seeman,
editors, Proceedings of Seventh International Conference on DNA based
Computers (DNA7), LNCS 2340, pages 290--299, 2002.
\item M. Sakthi Balan, K. Krithivasan, Binding-Blocking Automata,
Preliminary proceedings of International Meeting on DNA Based
Computers, M. Hagiya and A. Ohuchi (Eds.), 2002, pp. 327.
\item M. Sakthi Balan, Watson-Crick Distributed Automata, SIAM
Discrete Mathematics Conference, San Diego, USA, 2002.
\item M. Sakthi Balan, K. Krithivasan, Normal-Forms of
Blocking-Binding Automata, Unconventional Models of Computing,
published as CDMTCS Research Report at the University of Auckland,
CDMTCS-195, C.S. Calude and M.J. Dinneen and F. Peper (Eds.), Japan,
2002, pp. 3.
\item M. Sakthi Balan, Complexity Issues in Binding-Blocking Automata,
Pre-proceedings of International Workshop on Descriptional Complexity
of Formal Systems, University of Western Ontario, London, Ontario,
Canada, Aug 21-24, 2002, J. Dassow, M. Hoeberechts, H. J\"{u}rgensen and
D. Wotschke (Eds.), pp. 43--54.
\item M. Sakthi Balan, Parallel Communicating Pushdown Automata with
Filters, ed. J. Dassow and D. Wotschke, Proceedings of Third
International Workshop on Descriptional Complexity of Automata,
Grammars and Related Structures, Vienna, Austria, July 21-22, 2001,
Preprint Nr. 16 of the Fakultat fur Informatik,
Otto-von-Guericke-Universitat, Magdeburg, pages 167--175, 2001.
\item K. Krithivasan and M. Sakthi Balan, Distributed Processing in
Deterministic PDA, In Proceedings of the International Workshop on
Grammar Systems, Austria, ed. R.Freund and A.Kelemenova, pp.  127--145,
2000.
\item K. Krithivasan and M. Sakthi Balan, Some properties of Array
Contextual Grammars, presented in National Seminar on Discrete
Mathematics and Applications, 2000.
\end{enumerate}

{\large \bf Technical Reports}

\begin{enumerate}
\item M. Sakthi Balan, H. J\"{u}rgensen, Peptide Computing - Universality
and Theoretical Model, preprint 1/2006, ISSN 0946-7580, Universitat
Potsdam, Germany, May 2006.
\item M. Sakthi Balan, H. J\"{u}rgensen, K. Krithivasan, Peptide
  Computing: A Survey, preprint 4/2005, ISSN 0946-7580,
  Universit\"{a}t Potsdam, Germany, Nov 2005.
\item M. Sakthi Balan, H. J\"{u}rgensen, On the Universality of
Peptide Computing, preprint 6/2005, ISSN 0946-7580, Universit\"{a}t
Potsdam, Germany.
\end{enumerate}

{\large \bf Manuscripts/Communicated Papers}

\begin{enumerate}
\item M. Sakthi Balan, K. Krithivasan, Binding-Blocking Automata, communicated.
\item M. Sakthi Balan, H. J\"{u}rgensen, Non-determinism in Peptide Computing, in preparation.
\item M. Sakthi Balan, K. Krithivasan, Variants of Binding-Blocking
Automata, manusript.

\end{enumerate}




\noindent {\large \bf Extra Curricular Activities}

\vspace{0.1cm}

\begin{itemize}
\item Co-ordinated the hosting of the following international
conferences at IIT Madras

\vspace{0.1cm}

\begin{center}
\begin{tabular}{|c|c|}\hline
Conference & Date \\ \hline
FST\&TCS & 17-19 Dec 1998\\
FST\&TCS & 13-15 Dec 1999\\
ISAAC & 16-18 Dec 1999\\
INDOCRYPT & 16-20 Dec 2001\\ \hline
\end{tabular}
\end{center}

\vspace{0.1cm}
\item Research Scholars Representative, Department of Computer Science
\& Engineering, IIT Madras, 2001-2002
\item Class Representative - MSc Maths: 1995-1997

\end{itemize}

\vspace{0.1cm}

\noindent {\large \bf Extras/Hobbies}

\vspace{0.1cm}

\begin{itemize}
\item Carnatic Classical Music Singer - student of Late Prof. Palghat Shri. K.V.Narayanaswamy. 
\item Passed Higher Grade Examinations in Music (Vocal) conducted by the Government of Tamil Nadu. 
\item Participated in Music Competitions and won many prizes. 
\item Was a member of University Cultural Team in 1995-1997
\item Playing chess and analyzing chess games.
\end{itemize}


\end{document}

